% Copyright 2008 2009 2010 2011 Douglas Wikstrom

% This file is part of Vfork.

% Vfork is free software: you can redistribute it and/or modify
% it under the terms of the GNU Lesser General Public License as
% published by the Free Software Foundation, either version 3 of the
% License, or (at your option) any later version.

% Vfork is distributed in the hope that it will be useful, but
% WITHOUT ANY WARRANTY; without even the implied warranty of
% MERCHANTABILITY or FITNESS FOR A PARTICULAR PURPOSE.  See the GNU
% Lesser General Public License for more details.

% You should have received a copy of the GNU Lesser General Public
% License along with Vfork.  If not, see
% <http://www.gnu.org/licenses/>.

\documentclass[11pt]{article}
\usepackage{vtm}
%\usepackage{tex4ht}

% For not wasting space in submission
\addtolength{\voffset}{-1.5cm}
\addtolength{\textheight}{3cm}

\addtolength{\hoffset}{-1cm}
\addtolength{\textwidth}{2.0cm}


%%%%%%%%%%%%%%%%%%%%%%%%%%%%%%%%%%%%%%%%%%%%%%%%%%%%%%%


\title{{\Huge DRAFT}\\$\quad$\\How to Implement a Stand-alone Verifier \\for the Vfork Mix-Net}

\author{Douglas Wikstr{\"o}m\\
KTH Stockholm, Sweden\\
\texttt{dog@csc.kth.se}}

% \author{Douglas Wikstr{\"o}m}
% \institute{
%  CSC KTH Stockholm, Sweden\\
%   \email{dog@csc.kth.se}
% }


\pagestyle{empty}

\begin{document}

%\Configure{$}{\PicMath}{\EndPicMath}{}


\maketitle

\thispagestyle{empty}

\begin{abstract}

  Vfork, \url{http://www.mixnet.org}, is a free and open
  source implementation of an El Gamal based mix-net which optionally
  uses the Fiat-Shamir heuristic to produce a universally verifiable
  proof of correctness during the execution of the protocol. This
  document gives a detailed description of this proof targeting
  implementers of stand-alone verifiers.

\end{abstract}

\begin{quote}

\subsection*{Protocol Version 0.1}

This document covers version 0.1 of the protocol. The Vfork
package version is independent of the protocol version.


\vfill

\subsection*{Help Us Improve This Document}

This document is a draft. The most recent version of this
document can be found at:
\url{http://www.mixnet.org/mixnet/vmnv-0.1.pdf}.

Please help us improve the quality of this document by reporting
errors, omissions, and suggestions to
\texttt{dog@csc.kth.se}.


\end{quote}

\newpage

\begin{spacing}{0.9}

  \tableofcontents

\end{spacing}

\clearpage

\pagestyle{plain}
\setcounter{page}{1}


\section{Introduction}\label{sect:introduction}

The \veri mix-net~\cite{vmn} can optionally be executed with
Fiat-Shamir proofs of correctness, i.e., non-interactive
zero-knowledge proofs in the random oracle model. These proofs end up
in a special \emph{proof directory} along with all intermediate
results published on the bulletin board during the execution. The
proofs and the intermediate results allows anybody to verify the
correctness of an execution as a whole, i.e., that the joint public
key, the input ciphertexts, and the output plaintexts are related as
defined by the protocol and the public parameters of the
execution. The goal of this document is to give a detailed description
of how to implement a stand-alone algorithm for verifying the complete
contents of the proof directory.


\section{Background}

Before we delve into the details of how to implement a verifier, we
recall the \elgamal cryptosystem and briefly describe the mix-net
implemented in \veri (in the case where the Fiat-Shamir heuristic is
used to construct non-interactive zero-knowledge proofs).

\subsection{The \elgamal~Cryptosystem}\label{sect:elgamal}

The \elgamal cryptosystem~\cite{elgamal} is defined over a group $\Gq$
of prime order $\q$. A secret key $\x\in\zedq$ is sampled randomly,
and a corresponding public key $(\g,\y)$ is defined by $\y=\g^{\x}$,
where $\g$ is (typically) the standard generator in the underlying
group $\Gq$. To encrypt a plaintext $\mess\in\Gq$, a random exponent
$\sand\in\zedq$ is chosen and the ciphertext is computed as
$\Enc_{(\g,\y)}(\mess,s)=(\g^{\sand},\y^{\sand}\mess)$. A plaintext
can then be recovered from a ciphertext $(\uciph,\vciph)$ as
$\Dec_{\x}(\uciph,\vciph)=\uciph^{-\x}\vciph=\mess$. To encrypt an
arbitrary string of bounded length $t$ we also need an injection
$\{0,1\}^{t}\to\Gq$, which can be efficiently computed and inverted.

\paragraph{Homomorphic.}

The cryptosystem is homomorphic, i.e., if
$(\uciphi{1},\vciphi{1})=\Enc_{(\g,\y)}(\messi{1},\sandi{1})$ and
$(\uciphi{2},\vciphi{2})=\Enc_{(\g,\y)}(\messi{2},\sandi{2})$ are two
ciphertexts, then their element-wise product
$(\uciphi{1}\uciphi{2},\vciphi{1}\vciphi{2})=\Enc_{(\g,\y)}(\messi{1}\messi{2},\sandi{1}+\sandi{2})$
is an encryption of $\messi{1}\messi{2}$. If we set $\messi{2}=1$,
then this feature can be used to \emph{re-encrypt} a ciphertext
without knowledge of the randomness used to form
$(\uciphi{1},\vciphi{1})$. To see this, note that for every fixed
$\sandi{1}$ and random $\sandi{2}$, $\sandi{1}+\sandi{2}$ is randomly
distributed in $\zedq$.

\paragraph{Distributed Key Generation.}

The \elgamal cryptosystem also allows efficient protocols for
distributed key generation and distributed decryption of ciphertexts
by $\Mk$ parties. The $l$th party generates its own secret key
$\xj{l}$ and defines a (partial) public key $\yj{l}=\g^{\xj{l}}$. In
addition to this, the parties jointly run a protocol that verifiably
secret shares the secret key $\xj{l}$ such that a threshold $\thres$
of the parties can recover it in the event that the $l$th party fails
to do its part correctly in the joint decryption of ciphertexts. The
details \cite{feld,gennaro} of the verifiable secret sharing scheme
are not important in this document. The joint public key is then
defined as $(\g,\y)$, where $\y=\prod_{l=1}^{\Mk}\yj{l}$. Note that
the corresponding secret key is defined by
$\x=\sum_{l=1}^{\Mk}\xj{l}$.

To jointly decrypt a ciphertext $(\uciph,\vciph)$, the $l$th party
publishes a \emph{partial} decryption factor $\decrfactj{l}$ computed
as $\PDec_{\xj{l}}(\uciph,\vciph)=\uciph^{\xj{l}}$ and proves using a
zero-knowledge proof that it computed the decryption factor correctly
relative to its public key $\yj{l}$. If the proof is rejected, then
the other parties recover the secret key $\xj{l}$ of the $l$th party
and perform its part of the joint decryption in the open. Then the
decryption factors can be combined to a joint decryption factor
$\decrfact=\prod_{l=1}^{\Mk}\decrfactj{l}$ such that
$\PDec_{\x}(\uciph,\vciph)=\decrfact$. The ciphertext can then be
trivially decrypted as
$\TDec((\uciph,\vciph),\decrfact)=\vciph/\decrfact=m$.

\paragraph{A Generalization and Useful Notation.}

Using a simple hybrid argument it is easy to see that a longer
plaintext $\mess=(\messi{1},\ldots,\messi{t})\in\Gq^t$ can be
encrypted by encrypting each component independently, as
$\big(\Enc_{(\g,\y)}(\messi{1},\sandi{1}),\ldots,\Enc_{(\g,\y)}(\messi{t},\sandi{t})\big)$
where $\sand=(\sandi{1},\ldots,\sandi{t})$ is an element of the
product ring $\RSpace=\zedq^t$ or randomizers.

In our setting it is more convenient to simply view the cryptosystem
as defined for elements in the product group $\MSpace=\Gq^t$ of
plaintexts directly, i.e., we define encryption as
$\Enc_{(\g,\y)}(\mess,\sand)=(\g^{\sand},\y^{\sand}\mess)$, where
$\sand$ is an element in the product ring $\RSpace=\zedq^t$. Thus, the
ciphertext belongs to the ciphertext space
$\CSpace=\MSpace\times\MSpace$. Here exponentiation is distributed
element-wise, e.g.,
$\g^{\sand}=(\g^{\sandi{1}},\ldots,\g^{\sandi{t}})$ and multiplication
is defined element-wise. Decryption and computation of decryption
factors can be defined similarly.

We remark that the \veri mix-net is based on this generalization of
the \elgamal cryptosystem. Perhaps future versions of this document
will describe this general setting, but in the current version we
focus on the basic case where the randomizer ring is $\RSpace=\zedq$,
the group of plaintexts is $\MSpace=\Gq$, and the group of ciphertexts
is $\CSpace=\MSpace\times\MSpace=\Gq\times\Gq$.


\subsection{A Mix-Net Based on the \elgamal~Cryptosystem}\label{sect:mixnet}

We use the re-encryption approach of Sako and Kilian~\cite{saki}
combined with the pre-computation technique in
Wikstr{\"o}m~\cite{wik09b} and the proof of a shuffle of Terelius and
Wikstr{\"o}m~\cite{TW10}. The choice of proof of a shuffle is mainly
motivated by the fact that all other efficient proofs of shuffles are
patented. We use the batching technique of Bellare et
al.~\cite{BeGaRa} to speed up the proofs needed during distributed
decryption.

The $\Mk$ mix-servers first run a distributed key generation protocol
such that for each $1\leq l\leq\Mk$, the $l$th mix-server has a public
key $\yj{l}$ and a corresponding secret key $\xj{l}\in\zedq$ which is
verifiably secret shared among all $\Mk$ mix-servers such that any set
of $\thres$ parties can recover $\xj{l}$, but no smaller subset learns
anything about $\xj{l}$. Then they define a joint public key
$(\g,\y)$, where $\y=\prod_{l=1}^{\Mk}\yj{l}$, to be used by senders.

There are $\PN$ senders. The $i$th sender encrypts its message
$\messi{i}\in\MSpace$ by picking $\sandi{i}\in\RSpace$ randomly and
computing a ciphertext
$\ciphji{0}{i}=\Enc_{(\g,\y)}(\messi{i},\sandi{i})$. To preserve
privacy, the sender must also prove that it knows the plaintext
$\messi{i}$ of its ciphertext. This can be ensured in different ways,
but it is of no concern in this document, since we only verify the
\emph{correctness} of an execution (and not privacy).

Recall that a non-interactive proof allows a prover to convince a
verifier that a given statement is true by sending a single
message. The verifier then either accepts the proof as valid or
rejects it as invalid. In this context a proof is said to be
zero-knowledge if, loosely, it does not reveal anything about the
witness of the statement known by the prover.

The mix-servers now form a list
$\List{0}=(\ciphji{0}{i})_{i\in[1,\PN]}$ of all the ciphertexts. Then
the $j$th mix-server proceeds as follows for $l=1,\ldots,\thres$:
\begin{itemize}

\item If $l=j$, then it re-encrypts each ciphertext in $\List{l-1}$,
  permutes the result and publishes this as $\List{l}$. More
  precisely, it chooses $\randji{l}{i}\in\RSpace$ and a permutation
  $\perm$ randomly and outputs
  $\List{l}=(\ciphji{l}{i})_{i\in[1,\PN]}$, where
  \begin{align}
    \ciphji{l}{i}=\ciphji{l-1}{\perm(i)}\Enc_{(\g,\y)}(1,\randji{l}{\perm(i)})\enspace.\label{eq:reencrypt}
  \end{align}
  Then it publishes a non-interactive zero-knowledge proof of
  knowledge $(\proofC{l},\proofR{l})$ of all the
  $\randji{l}{i}\in\zedq$ and that they satisfy (\ref{eq:reencrypt}).

\item If $l\neq j$, then it waits until the $l$th mix-server publishes
  $\List{l}$ and a non-interactive zero-knowledge proof of knowledge
  $(\proofC{l},\proofR{l})$. If the proof is rejected, then $\List{l}$
  is set equal to $\List{l-1}$.

\end{itemize}

Finally, the mix-servers jointly decrypt the ciphertexts in
$\List{\thres}$ as described in \sect{elgamal}. More precisely, the
$l$th mix-server computes
$\decrfactj{l}=\PDec_{\xj{l}}(\List{\thres})$ element-wise and gives a
non-interactive zero-knowledge proof $(\dproofC{l},\dproofR{l})$ that
$\decrfactj{l}$ was computed correctly. If the proof is rejected, then
$\xj{l}$ is recovered using the verifiable secret sharing scheme and
$\decrfactj{l}=\PDec_{\xj{l}}(\List{\thres})$ is computed in the
open. Then the output of the mix-net is computed as
$\TDec(\List{\thres},\prod_{l=1}^{\Mk}\decrfactj{l})$, where the
product and $\TDec(\cdot,\cdot)$ are taken element-wise.


\subsection{Outline of the Verification Algorithm}\label{sect:outlineverify}

We give a brief outline of the verification algorithm that checks that
intermediate results of an execution and all the zero-knowledge proofs
are consistent.

\begin{enumerate}

\item Check that the partial public keys are consistent with the
  public key used by senders to encrypt their messages, i.e., check
  that $\y=\prod_{l=1}^{\Mk}\yj{l}$. If not, then \reject.

\item Check that each mix-server re-encrypted and permuted the
  ciphertexts in its input or was ignored in the processing, i.e., for
  $l=1,\ldots,\thres$:
  \begin{itemize}

  \item If $(\proofC{l},\proofR{l})$ is not a valid proof of knowledge
    of exponents $\randji{l}{i}$ and a permutation $\perm$ such that
    $\ciphji{l}{i}=\ciphji{l-1}{\perm(i)}\Enc_{(\g,\y)}(1,\randji{l}{\perm(i)})$,
    then set $\List{l}=\List{l-1}$.

  \end{itemize}

\item Check that each party computed its decryption factors correctly
  or its secret key was recovered and its decryption factors computed
  openly, i.e., check that for $l=1,\ldots,\Mk$:
  \begin{itemize}

  \item If $\xj{l}$ was recovered such that $\yj{l}=g^{\xj{l}}$, then
    set $\decrfactj{l}=\PDec_{\xj{l}}(\List{l})$.

  \item Otherwise, if $(\dproofC{l},\dproofR{l})$ is not a valid proof
    that $\decrfactj{l}=\PDec_{\xj{l}}(\List{l})$ and
    $\yj{l}=\g^{\xj{l}}$, where $\decrfactj{l}$ are the decryption
    factors computed by the $l$th mix-server, then \reject.

  \end{itemize}

\item Check if the output of the mix-net is
  $\TDec(\List{\thres},\prod_{l=1}^{\Mk}\decrfactj{l})$. If not, then
  \reject and otherwise \accept.

\end{enumerate}

\paragraph{On the Use of Pre-computation in \veri.}

To speed up the mixing process, the \veri mix-net allows most of the
computations to be done before any ciphertexts have been received. To
achieve this, an upper bound $\PNmax$ on the number of ciphertexts is
established and then the ciphertexts $\Enc_{(\g,\y)}(1,\randji{l}{i})$
are pre-computed. This turns costly exponentiations into cheap
multiplications in the mixing phase. Furthermore, the non-interactive
proof of knowledge $(\proofC{l},\proofR{l})$ is split into a
commitment $\permcommj{l}$ of a permutation $\perm$ of $\PNmax$
elements, a proof of a shuffle of commitments
$(\posproofC{l},\posproofR{l})$ that this was formed correctly, and a
commitment-consistent proof of knowledge
$(\ccposproofC{l},\ccposproofR{l})$ of the exponents $\randji{l}{i}$
such that (\ref{eq:reencrypt}) holds. In other words, we may think of
the complete proof of a shuffle as
\begin{align*}
  (\proofC{l},\proofR{l})=\big((\permcommj{l},\posproofC{l},\ccposproofC{l}),(\posproofR{l},\ccposproofR{l})\big)\enspace.
\end{align*}
Only the proof $(\ccposproofC{l},\ccposproofR{l})$ is computed (or
verified by the other mix-servers) in the mixing phase, and this is
very efficient compared to the first part of the proof.

If the actual number $\PN$ of ciphertexts is smaller than $\PNmax$,
then the permutation commitment $\permcommj{l}$ is ``shrunk'' before
it is used. To do this, the prover simply identifies a suitable subset
of size $\PN$ of the elements of $\permcommj{l}$ (see \sect{verify}
for details).

We remark that when $\PNmax$ equals the actual number of ciphertexts
$\PN$, then the overhead cost of dividing the work in this way instead
of having single proof is small compared to the cost of a single
combined proof.


\section{How to Write a Verifier}

As explained in \sect{mixnet}, an execution of the mix-net is correct
if: (1) the joint public key used by senders to encrypt their messages
is consistent with the partial keys of the mix-servers, (2) the joint
public key was used to re-encrypt and permute the input ciphertexts,
and (3) the secret keys corresponding to the partial keys were used to
compute decryption factors. To turn the outline of the verification
algorithm in \sect{outlineverify} into an actual verification
algorithm, we must specify: all the parameters of the execution, the
representations of all arithmetic objects, the zero-knowledge proofs,
and how the Fiat-Shamir heuristic is applied.

\subsection{List of Manageable Sub-Tasks}

We divide the problem into a number of more manageable sub-tasks and
indicate which steps depend on previous steps to simplify the
distribution of the implementation work.

\begin{enumerate}

\item\label{step:bytetree}\textbf{Byte Trees.} All of the mathematical
  and cryptographic objects are represented as so called \emph{byte
    trees}. \sect{bytetrees} describes this simple and
  language-independent byte-oriented format.

\item\label{step:primitives}\textbf{Cryptographic Primitives.} We need
  concrete implementations of hash functions, pseudo-random generators,
  and random oracles, and we must define how these objects are
  represented. This is described in \sect{primitives}.

\item\label{step:arithmetic}\textbf{Arithmetic Library.} An arithmetic
  library is needed to compute with algebraic objects, e.g., group
  elements and field elements. These objects also need to be converted
  to and from their representations as byte trees. \sect{arithmetic}
  describes how this is done.

\item\label{step:infofile}\textbf{Protocol Info Files.}  Some of
  the public parameters, e.g., auxiliary security parameters, must be
  extracted from an XML encoded protocol info file before any
  verification can take place. \sect{infofiles} describes the format
  of this file and which parameters are extracted.

\item\label{step:basicverify}\textbf{Verifying Fiat-Shamir Proofs.}
  The tests performed during verification are quite
  complex. \sect{basicverify} explains how to implement these tests.

\item\label{step:verify}\textbf{Verification of a Complete Execution.}
  \sect{verify} combines all of the above steps into a single
  verification algorithm.

\end{enumerate}

\subsection{How to Divide the Work}

\step{bytetree} does not depend on any other step. \step{primitives}
and \step{arithmetic} are independent from the other steps except for
how objects are encoded to and from their representation as byte
trees. \step{infofile} can be divided into the problem of parsing an
XML file and then interpreting the data stored in each XML block. The
first part is independent of all other steps, and the second part
depends on \step{bytetree}, \step{primitives} and
\step{arithmetic}. \step{basicverify} depends on \step{bytetree},
\step{primitives}, and \step{arithmetic}, but not on \step{infofile},
and it may internally be divided into separate tasks. \step{verify}
depends on all previous steps.


\section{Byte Trees}\label{sect:bytetrees}

We use a byte-oriented format to represent objects on file and to turn
them into arrays of bytes that can be input to a hash function. The
goal of this format is to be as simple as possible.

\subsection{Definition}

A byte tree is either a \emph{leaf} containing an array of bytes, or a
\emph{node} containing other byte trees. We write \leaf{$d$} for a
leaf with a byte array $d$ and we write \node{$b_1,\ldots,b_l$} for a
node with children $b_1,\ldots,b_l$. Complex byte trees are then easy
to describe.

\begin{example}\label{exam:bytetree}
  The byte tree containing the data \hex{AF}, \hex{03E1}, and
  \hex{2D52} (written in hexadecimal) in three leaves, where the first
  two leaves are siblings is
\begin{align*}
  \node{\node{\leaf{\hex{AF}}, \leaf{\hex{03E1}}}, \leaf{\hex{2D52}}}\enspace.
\end{align*}
\end{example}

\subsection{Representation as an Array of Bytes}

We use $\nbytes{k}{n}$ as a short-hand to denote the two's-complement
representation of $n$ in big endian byte order truncated to the $k$
least significant bytes. We also use hexadecimal notation for
constants, e.g., $\hex{0A}$ means $\nbytes{1}{10}$.

A byte tree is represented as an array of bytes as follows.
\begin{itemize}

\item A leaf $\leaf{d}$ is represented as the concatenation of: a
  single byte $\hex{01}$ to indicate that it is a leaf, four bytes
  $\nbytes{4}{l}$, where $l$ is the number of bytes in $d$, and the
  data bytes $d$.

\item A node $\node{b_1,\ldots,b_l}$ is represented as the
  concatenation of: a single byte \hex{00} to indicate that it is a
  node, four bytes $\nbytes{4}{l}$ representing the number of
  children, and $\bytes{b_1}\mid\bytes{b_2}\mid\cdots\mid\bytes{b_l}$,
  where $\mid$ denotes concatenation and $\bytes{b_i}$ denotes the
  representation of the byte tree $b_i$ as an array of bytes.

\end{itemize}

\begin{example}[\exam{bytetree} contd.]
  The byte tree is represented as the following array of bytes.
  \begin{align*}
    &\hex{00\cutt00\cut00\cut00\cut02}\\
    &\quad\hex{00\cutt00\cut00\cut00\cut02}\\
    &\quad\quad\hex{01\cutt00\cut00\cut00\cut01\cutt AF}\\
    &\quad\quad\hex{01\cutt00\cut00\cut00\cut02\cutt03\cut E1}\\
    &\quad\hex{01\cutt00\cut00\cut00\cut02\cutt 2D\cut 52}
  \end{align*}
\end{example}


It is a good idea to hand (an upper bound of) the expected recursive
depth to a parser to let it fail in a controlled way if the input does
not represent a valid byte tree.

ASCII strings are converted to byte trees in the
natural way, i.e., a string $s$ (an array of bytes) is converted to
$\leaf{s}$.
\begin{example}
  The string $\code{"ABCD"}$ is represented by $\leaf{\hex{65666768}}$.
\end{example}
Sometimes we store byte trees as the hexadecimal encoding of their
representation as an array of bytes. We denote by $\tohex{a}$ the
hexadecimal encoding of an array of bytes. We abuse notation and
simply write $a$ instead of $\bytes{a}$ when the context is clear,
e.g., if $\hash$ is a hash function we write $\hash(a)$ instead of
$\hash(\bytes{a})$.

\subsection{Backus-Naur Grammar.}

The above description should suffice to implement a parser for byte
trees, but for completeness we give their Backus-Naur grammar. We
denote the $n$-fold repetition of a symbol
$\langle\textit{rule}\rangle$ by $n\langle\textit{rule}\rangle$. The
grammar then consists of the following rules for
$n=0,\ldots,2^{31}-1$, where $\mid$ denotes choice.
\begin{align*}
  \langle\textit{bytetree}\rangle&::=\langle\textit{leaf}\rangle\mid\langle\textit{node}\rangle\\
  \langle\textit{leaf}\rangle&::=\hex{01}\langle\textit{uint}_n\rangle\langle\textit{data}_n\rangle\\
  \langle\textit{node}\rangle&::=\hex{00}\langle\textit{uint}_n\rangle\langle\textit{bytetrees}_n\rangle\\
  \langle \textit{uint}_n\rangle&::=\nbytes{4}{n}\\
  \langle \textit{data}_n\rangle&::=n\langle\textit{byte}\rangle\\
  \langle\textit{bytetrees}_n\rangle&::=n\langle\textit{bytetree}\rangle\\
  \langle \textit{byte}\rangle&::=\hex{00}\mid \hex{01}\mid
  \hex{02}\mid\cdots\mid \hex{FF}
\end{align*}


\section{Cryptographic Primitives}\label{sect:primitives}

For our cryptographic library we need hash functions, pseudo-random
generators, and random oracles derived from these. We must also define
their functionality.

\subsection{Hash Functions}\label{sect:hashfunctions}

\veri allows an arbitrary hash function to be used, but in this
document we restrict our attention to the SHA-2 family \cite{sha},
i.e., SHA-256, SHA-384, and SHA-512. Before the winner of the SHA-3
competition has been announced, we see no reason to use any other
cryptographic hash function to instantiate the random oracle model. We
use the following notation.
\begin{itemize}

% \item $\bt{\hash}$ denotes the byte tree representation\footnote{It
%     may seem overly complicated to, e.g., use a byte tree
%     representation of the string \code{"SHA-256"} to configure the
%     mix-net to use this hash function, but \veri can be configured
%     with hash functions that cannot be represented as short strings.}
%   of a hash function $\hash$. This is defined as one of the byte trees
%   $\leaf{\code{"SHA-256"}}$, $\leaf{\code{"SHA-384"}}$, or
%   $\leaf{\code{"SHA-512"}}$ depending on which hash function is
%   represented.

% \item$\Hashfunction(\bt{\hash})$ -- Creates a hash function $\hash$
%   from its byte tree representation $\bt{\hash}$.

\item$\Hashfunction(s)$ -- Creates a hashfunction from
  \code{"SHA-256"}, \code{"SHA-384"}, or \code{"SHA-512"}.

\item$\hash(d)$ -- Denotes the hash digest of the byte array $d$.

\item$\outlen{\hash}$ -- Denotes the number of bits in the output of
  the hash function $\hash$.

\end{itemize}

% \begin{example}
%   If $\hash=\Hashfunction(\leaf{\code{"SHA-256"}})$ and $d$ is a byte
%   tree then $\hash(d)$ denotes the hash digest of the array of bytes
%   representing the byte tree as computed by SHA-256, and
%   $\outlen{\hash}$ equals $256$.
% \end{example}

\begin{example}
  If $\hash=\Hashfunction(\code{"SHA-256"})$ and $d$ is a byte tree
  then $\hash(d)$ denotes the hash digest of the array of bytes
  representing the byte tree as computed by SHA-256, and
  $\outlen{\hash}$ equals $256$.
\end{example}

\subsection{Pseudo-Random Generators}\label{sect:prg}

We need a pseudo-random generator (PRG) to expand a short challenge
string into a long ``random'' vector to use batching techniques in the
zero-knowledge proofs of \sect{basicverify}. \veri allows any
pseudo-random generator to be used, but in the random oracle model
there is no need to use a provably secure PRG. We consider a simple
construction based on a hash function $\hash$.

The PRG takes a seed $s$ of $\secp_{\hash}=\outlen{\hash}$ bits
as input. Then it generates a sequence of bytes $r_0\mid r_1\mid
r_2\mid\cdots$, where $\mid$ denotes concatenation, and $r_i$ is an
array of $\secp_{\hash}/8$ bytes defined by
\begin{align*}
  r_i=\hash(s\mid\nbytes{4}{i})
\end{align*}
for $i=0,1,\ldots,2^{31}-1$, i.e., in each iteration we hash the
concatenation of the seed and an integer counter (four bytes). It is
not hard to see that if $\hash(s\mid\cdot)$ is a pseudo-random
function for a random choice of the seed, then this is a provably
secure construction of a pseudo-random generator. We use the following
notation.
\begin{itemize}

% \item $\bt{\prg}$ -- Denotes the byte tree representation of an
%   instance $\prg$. This is simply defined as $\bt{\hash}$, i.e., the
%   byte tree representation of the underlying hash function.

% \item$\PRG(r)$ -- Creates an unseeded instance $\prg$ from its byte
%   tree representation $\bt{\prg}$ or from a hash function $\hash$,
%   i.e., $r$ is either of the form $\bt{\prg}$ or it is $\hash$.

\item$\PRG(\hash)$ -- Creates an unseeded instance $\prg$ from a hash
  function $\hash$.

\item$\seedlen{\prg}$ -- Denotes the number of seed bits needed as
  input by $\prg$.

\item$\prg(s)$ -- Denotes an array of pseudo-random bytes derived from
  the seed $s$. Strictly speaking this array is
  $2^{31}\secp_{\hash}$ bits long, but we simply write
  $(t_0,\ldots,t_l)=\prg(s)$, where $t_i$ is of a given bit length,
  instead of explicitly saying that we iterate the construction a
  suitable number of times and then truncate to the exact output
  length we want.

\end{itemize}


\subsection{Random Oracles}\label{sect:randomoracle}

We need a flexible random oracle that allows us to derive any number
of bits. We use a construction based on a hash function $\hash$. To
differentiate the random oracles with different output lengths, the
output length is used as a prefix in the input to the hash
function. The random oracle first constructs a pseudo-random generator
$\prg=\PRG(\hash)$ which is used to expand the input to the requested
number of bits. To evaluate the random oracle on input $d$ the random
oracle then proceeds as follows, where $\secp_{\mathit{out}}$ is the
output length in bits.
\begin{enumerate}

\item Compute $s=\hash(\nbytes{4}{\secp_{\mathit{out}}}\mid d)$, i.e.,
  compress the concatenation of the output length and the actual data
  to produce a seed $s$.

\item Let $a$ be the $\lceil\secp_{\mathit{out}}/8\rceil$ first
  bytes in the output of $\prg(s)$.

\item Set the $8-(\secp_{\mathit{out}}\bmod 8)$ first bits of $a$ to
  zero, and output the result.

\end{enumerate}
We remark that setting some of the first bits of the output to zero to
emulate an output of arbitrary bit length is convenient in our
setting, since it allows the outputs to be directly interpreted as
random integers of a given (nominal) bit length.

This construction is a secure implementation of a random oracle with
output length $\secp_{\mathit{out}}$ for any $\secp_{\mathit{out}}\leq
2^{31}\outlen{\hash}$ when $\hash$ is modeled as a random oracle and
the PRG of \sect{prg} is used. Note that it is unlikely to be a secure
implementation if a PRG is used which is provably secure under some
computational assumption.

There is no need to represent random oracles as byte trees. Thus, we
only need the following notation:
\begin{itemize}

\item$\RandomOracle(\hash,\secp_{\mathit{out}})$ -- Creates a random
  oracle with output length $\secp_{\mathit{out}}$ from the hash
  function $\hash$.

\item$\RO(d)$ -- Denotes the output of the random oracle $\RO$ on
  an input byte array $d$.
 
\end{itemize}

\section{Representations of Arithmetic Objects}\label{sect:arithmetic}

Every arithmetic object in \veri is represented as a byte tree. In
this section we pin down the details of these representations.
\begin{itemize}

\item\textbf{Integers.} A multi-precision integer $n$ is represented
  by $\leaf{\nbytes{k}{n}}$ for the smallest possible $k$.

  \begin{example}
    $263$ is represented by
    \hex{01\cutt00\cut00\cut00\cut02\cutt01\cut07}.
  \end{example}
  \begin{example}
    $-263$ is represented by
    \hex{01\cutt00\cut00\cut00\cut02\cutt FE\cut F9}.
  \end{example}

\item\textbf{Arrays of Booleans.} An array $(a_1,\ldots,a_l)$ of
  booleans is represented as $\leaf{b}$, where $b=(b_1,\ldots,b_l)$ is
  an array of bytes where $b_i$ equals $\hex{01}$ if $a_i$ is true and
  $\hex{00}$ otherwise.

  \begin{example}
    The array $(\True,\False,\True)$ is represented by
    $\leaf{\hex{01\cut00\cut01}}$.
  \end{example}
  \begin{example}
    The array $(\True,\True,\False)$ is represented by
    $\leaf{\hex{01\cut01\cut00}}$.
  \end{example}

\item\textbf{Field Elements.} An element $a$ in a prime order field
  $\zedq$ is represented by $\leaf{\nbytes{k}{a}}$, where $a$ is
  identified with its integer representative in $[0,q-1]$ and $k$ is
  the smallest possible $k$ such that $q$ can be represented as
  $\nbytes{k}{q}$. In other words, field elements are represented
  using fixed size byte trees, where the fixed size depends on the
  order of the field.

  \begin{example}
    $258\in\zed_{263}$ is represented by
    \hex{01\cutt00\cut00\cut00\cut02\cutt01\cut02}.
  \end{example}
  \begin{example}
    $5\in\zed_{263}$ is represented by
    \hex{01\cutt00\cut00\cut00\cut02\cutt00\cut05}.
  \end{example}

\item\textbf{Array of Field Elements.} An array $(a_1,\ldots,a_l)$ of
  field elements is represented by $\node{\bt{a_1},\ldots,\bt{a_l}}$,
  where $\bt{a_i}$ is the byte tree representation of $a_i$.

  \begin{example}
    The array $(1,2,3)$ of elements in $\zed_{263}$ is represented by:
    \begin{align*}
      &\hex{00\cutt00\cut00\cut00\cut03}\\
      &\quad\hex{01\cutt00\cut00\cut00\cut02\cutt00\cut01}\\
      &\quad\hex{01\cutt00\cut00\cut00\cut02\cutt00\cut02}\\
      &\quad\hex{01\cutt00\cut00\cut00\cut02\cutt00\cut03}
    \end{align*}
  \end{example}

\item\textbf{Product Ring Element.} An element $a=(a_1,\ldots,a_l)$ in
  a product ring is represented by $\node{\bt{a_1},\ldots,\bt{a_l}}$,
  where $\bt{a_i}$ is the byte tree representation of the component
  $a_i$. Note that this representation keeps information about the
  order in which a product group is formed intact (see the second
  example below).

  \begin{example}
    The element $(258,5)\in\zed_{263}\times\zed_{263}$ is represented
    by:
    \begin{align*}
      &\hex{00\cutt00\cut00\cut00\cut02}\\
      &\quad\hex{01\cutt00\cut00\cut00\cut02\cutt01\cut02}\\
      &\quad\hex{01\cutt00\cut00\cut00\cut02\cutt00\cut05}
    \end{align*}
  \end{example}
  \begin{example}
    The element
    $((258,6),5)\in(\zed_{263}\times\zed_{263})\times\zed_{263}$ is
    represented by:
    \begin{align*}
      &\hex{00\cutt00\cut00\cut00\cut02}\\
      &\quad\hex{00\cutt00\cut00\cut00\cut02}\\
      &\quad\quad\hex{01\cutt00\cut00\cut00\cut02\cutt01\cut02}\\
      &\quad\quad\hex{01\cutt00\cut00\cut00\cut02\cutt00\cut06}\\
      &\quad\hex{01\cutt00\cut00\cut00\cut02\cutt00\cut05}
    \end{align*}
  \end{example}

\item\textbf{Array of Product Ring Elements.} An array
  $(a_1,\ldots,a_l)$ of elements in a product ring, where
  $a_i=(a_{i,1},\ldots,a_{i,k})$, is represented by
  $\node{\bt{b_1},\ldots,\bt{b_k}}$, where $b_i$ is the array
  $(a_{1,i},\ldots,a_{l,i})$ and $\bt{b_i}$ is its representation as a
  byte tree.

  Thus, the structure of the representation of an array of ring
  elements mirrors the representation of a single ring element. This
  seemingly contrived representation turns out to be convenient in
  implementations.

  \begin{example}
    The array $\big((1,4),(2,5),(3,6)\big)$ of elements in
    $\zed_{263}\times\zed_{263}$ is represented as
    \begin{align*}
      &\hex{00\cutt00\cut00\cut00\cut02}\\
      &\quad\hex{00\cutt00\cut00\cut00\cut03}\\
      &\quad\quad\hex{01\cutt00\cut00\cut00\cut02\cutt00\cut01}\\
      &\quad\quad\hex{01\cutt00\cut00\cut00\cut02\cutt00\cut02}\\
      &\quad\quad\hex{01\cutt00\cut00\cut00\cut02\cutt00\cut03}\\
      &\quad\hex{00\cutt00\cut00\cut00\cut03}\\
      &\quad\quad\hex{01\cutt00\cut00\cut00\cut02\cutt00\cut04}\\
      &\quad\quad\hex{01\cutt00\cut00\cut00\cut02\cutt00\cut05}\\
      &\quad\quad\hex{01\cutt00\cut00\cut00\cut02\cutt00\cut06}
    \end{align*}
  \end{example}

\item\textbf{Modular Group.} A subgroup $\Gq$ of order $\q$ of the
  multiplicative group $\zed_{\p}^*$, where $\p>3$ is prime, with
  standard generator $\g$ is represented by the byte tree
  $\node{\bt{\p},\bt{\q},\bt{\g},\nbytes{4}{e}}$, where the integer
  $e\in\{0,1,2\}$ determines how a string is encoded into a group
  element. We denote by $\ModPGroup(b)$ the group recovered from such
  a byte tree $b$.

  We may only have $e=1$ if $\p=2\q+1$, and we may only have $e=2$ if
  $\p=t\q+1$ and the difference between the bit lengths of $\p$ and
  $\q$ is less than $2^{10}$.

  The three encoding schemes are defined as follows.
  \begin{itemize}

  \item If $e=0$, then at most 3 bytes can be encoded. Let $\hash$
    denote SHA-256.

    \begin{description}

    \item[Encode.] To encode a sequence $\mess$ of $0\leq
      b_{\mess}\leq3$ bytes into an element $a$, find an element
      $a\in\Gq$ such that the first $b_{\mess}+1$ bytes of
      $\hash(\bt{a})$ are of the form $\nbytes{1}{l}\mid\mess$ for
      some $l$ such that $l\bmod 4=b_{\mess}$.

    \item[Decode.] To decode an element $a\in\Gq$ into a message
      $\mess$, let $\nbytes{1}{l'}$ be the first byte of
      $\hash(\bt{a})$ and define $l=l'\bmod4$. Then let $\mess$ be the
      next $l$ bytes of $\hash(\bt{a})$.

    \end{description}

  \item If $e=1$, then at most $b=\lfloor(\secp-2)/8\rfloor-4$ bytes
    can be encoded, where $\secp$ is the bit length of $\p$.
    \begin{description}

    \item[Encode.] To encode a sequence $\mess$ of $0\leq
      b_{\mess}\leq b$ bytes as an element $a\in\Gq$ do as follows.
      If $b_{\mess}=0$, then set $\mess'=\hex{01}\mid\nbytes{b-1}{0}$
      and otherwise set
      $\mess'=\mess\mid\nbytes{b-b_{\mess}}{0}$. Then interpret
      $\nbytes{4}{b_{\mess}}\mid\mess'$ as a positive integer
      $k$. Then let $a$ be $k$ or $\p-k$ so that $a\in\Gq$.

    \item[Decode.] To decode an element $a\in\Gq$ to a message
      $\mess$, set $k$ equal to $a$ or $\p-a$ depending on if $a<\p-a$
      or not. Then interpret $k\bmod 2^{8(b+4)}$ as
      $\nbytes{4}{l}\mid\mess'$, where $\mess'$ is a sequence of $b$
      bytes. If $l<0$ or $l>b$, then set $l=0$. Then let $\mess$ be the
      $l$ first bytes of $\mess'$.

    \end{description}

  \item If $e=2$, then at most $b=\lfloor\secp/8\rfloor-\secp'-4$
    bytes can be encoded, where $\secp$ is the bit length of $\p$ and
    $\secp'=\lceil t/8\rceil+1$.
    \begin{description}

    \item[Encode.] To encode a sequence $\mess$ of $0\leq
      b_{\mess}\leq b$ bytes as an element $a\in\Gq$, interpret
      $\nbytes{4}{b_{\mess}}\mid\mess\mid\nbytes{b-b_{\mess}}{0}$ as
      an integer $k$. Then let $a$ be the smallest integer of the form
      $i2^{8(b+4)}+k$ in $\Gq$ for some non-negative $i$.

    \item[Decode.] To decode an element $a\in\Gq$ to a message
      $\mess$, interpret $a\bmod 2^{8(b+4)}$ as
      $\nbytes{4}{l}\mid\mess'$, where $\mess'$ is a sequence of $b$
      bytes. If $l<0$ or $l>b$, then set $l=0$. Then let $\mess$ be the
      $l$ first bytes of $\mess'$

    \end{description}

  \end{itemize}

\item\textbf{Element of Modular Group.} An element $a\in\Gq$, where
  $\Gq$ is a subgroup of order $\q$ of $\zed_{\p}^*$ for a prime $\p$
  is represented by $\leaf{\nbytes{k}{a}}$, where $a$ is identified
  with its integer representative in $[0,\p-1]$ and $k$ is the
  smallest integer such that $\p$ can be represented as
  $\nbytes{k}{\p}$.

  \begin{example}
    Let $\Gq$ be the subgroup of order $\q=131$ in
    $\zed_{263}^*$. Then $258\in\Gq$ is represented by
    \hex{01\cutt00\cut00\cut00\cut02\cutt01\cut02}.
  \end{example}

\item\textbf{Array of Group Elements.} An array $(a_1,\ldots,a_l)$ of
  group elements is represented as $\node{\bt{a_1},\ldots,\bt{a_l}}$,
  where $\bt{a_i}$ is the byte tree representation of $a_i$.

\item\textbf{Product Group Element.} An element $a=(a_1,\ldots,a_l)$
  in a product group is represented by
  $\node{\bt{a_1},\ldots,\bt{a_l}}$, where $\bt{a_i}$ is the byte tree
  representation of $a_i$. This is similar to the representation of
  product rings.

\item\textbf{Array of Product Group Elements.} An array
  $(a_1,\ldots,a_l)$ of elements in a product group, where
  $a_i=(a_{i,1},\ldots,a_{i,k})$, is represented by
  $\node{\bt{b_1},\ldots,\bt{b_k}}$, where $b_i$ is the array
  $(a_{1,i},\ldots,a_{l,i})$ and $\bt{b_i}$ is its representation as a
  byte tree. This is similar to the representation of arrays of
  elements in a product ring.


\end{itemize}


\paragraph{Deriving Group Elements from Random Bits.}

In \sect{independent} we need to derive group elements from the output
of a pseudo-random generator $\prg$. (Strictly speaking we use $\prg$
as a random oracle here, but this is secure due to how it is defined.)
Exactly how this is done depends on the group and an auxiliary
security parameter $\secpr$. We denote this by
\begin{align*}
  \h=(\hi{0},\ldots,\hi{\PNmax-1})=\random{\Gq}{\PNmax,\prg(s),\secpr}
\end{align*}
and describe how this is defined for each type of group below. The
auxiliary security parameter determines the statistical distance in
distribution between a randomly chosen group element and the element
derived as explained below if we assume that the output of the PRG is
truly random.

We stress that it must be infeasible to find a non-trivial
representation of the unit of the group in terms of these generators,
i.e., it should be infeasible to find $e,e_0,\ldots,e_{\PNmax-1}$ not
all zero such that $\g^{e}\prod_{i=0}^{\PNmax-1}\hi{i}^{e_i}=1$. In
particular, it is not acceptable to derive exponents
$x_0,\ldots,x_{\PNmax-1}\in\zedq$ and then define $\hi{i}=g^{x_i}$.

\begin{itemize}

\item\textbf{Modular Group.} Let $\Gq$ be the subgroup of order $\q$
  of the multiplicative group $\zed_{\p}$, where $\p>3$ is prime. Then
  an array $(\hi{0},\ldots,\hi{\PNmax-1})$ in $\Gq$ is derived as
  follows from a seed $s$.
  \begin{enumerate}

  \item Let $\secp_{\p}$ be the bit length of $\p$.

  \item Let $(t_0,\ldots,t_{\PNmax-1})=\prg(s)$, where
    $t_i\in\{0,1\}^{8\lceil(\secp_{\p}+\secpr)/8\rceil}$ is
    interpreted as a \emph{non-negative} integer.

  \item Set $t_i'=t_i\bmod 2^{\secp_{\p}+\secpr}$ and let
    $\hi{i}=(t_i')^{(\p-1)/\q}\bmod\p$.

  \end{enumerate}
  In other words, for each group element $\hi{i}$ we first extract the
  minimum number of complete bytes
  $\lceil(\secp_{\p}+\secpr)/8\rceil$.  Then we reduce the number of
  bits to exactly $\secp_{\p}+\secpr$. Finally, we map the resulting
  integer into $\Gq$ using the canonical homomorphism.\footnote{This
    construction makes more sense if one considers an
    implementation. It is natural to implement a routine that derives
    an array of non-negative integers $t_i'$ of a given bit length
    $\secp_{\p}+\secpr$ as explained above. Group elements are then
    derived from such an array in the natural way by mapping the
    integers into $\Gq$.}

\end{itemize}


\section{Marshalling Group Descriptions}
\label{sect:unmarshal}

When groups convert themselves to byte trees in \veri, they do not
store the name of the Java class of which they are instances. Thus, to
recover a group from such a representation, information about the
class must be otherwise available. When we need to recover a group
from file, we store not only its internal state, but also its Java
class name. Then Java reflection is used to instantiate the right
class with the given internal state. This gives a reasonably language
independent format, since the names of classes can always be
tabulated.

%  The
% details of the scheme is best explained by an example.
% \begin{example}\label{exam:sha2}
%   A wrapper of the SHA-2 family of hash functions is provided by the
%   Java class \code{mixnet.crypto.HashfunctionHeuristic}. The
%   internal state of an instance of this class simply consists of the
%   name of the underlying algorithm, e.g., the string
%   \code{"SHA-256"}. As explained in \sect{hashfunctions}, such an
%   instance is converted to a byte tree $\leaf{\code{"SHA-256"}}$. The
%   complete byte tree is then
%   \begin{align*}
%     \node{\leaf{\code{"mixnet.crypto.HashfunctionHeuristic"}},\leaf{\code{"SHA-256"}}}\enspace.
%   \end{align*}
% \end{example}
% Fortunately, we need only parse byte trees of this more elaborate
% types for a few of our classes. We have the following mapping of Java
% class names and the notation used here.

We use the following notation to marshal and unmarshal
objects.
\begin{itemize}

\item $\marshal{a}$ -- Denotes
  $\node{\leaf{\code{"JavaClassName"}},\bt{a}}$, where $a$ in the
  execution of the \veri mix-net would be an instance of
  \code{JavaClassName} and $\bt{a}$ is the byte tree representation of
  $a$.

\item $\unmarshal{b}$ -- Denotes the instance $a$ such that in an
  execution of the \veri mix-net we would have $b=\marshal{a}$.

\end{itemize}
The group description stored in the protocol info file is represented
by the hexadecimal encoding of its representations as an array of
bytes prepended with a brief human oriented ASCII comment describing
the group. The end of the comment is indicated by double colons. If
$a$ is such an hexadecimal encoding, we simply write $\unmarshal{a}$
and assume that the comment is removed before the string is decoded
into an array of bytes, which in turn is decoded to a group.

The notation used in this document for groups correspond to Java
classes in \veri as follows.

\vspace{0.3cm}
\noindent
\begin{tabular}{l|l}
Notation & Java Classname in Vfork\\
\hline
$\ModPGroup$ & \code{mixnet.arithm.ModPGroup}\\
\end{tabular}


% \begin{example}[\exam{sha2} contd.]
%   The hexadecimal encoding of a hash function including a leading
%   comment briefly describing the payload appears as follows.
% \begin{lstlisting}[frame=single,language=xml,
% basicstyle=\tt\xmlfontsize]
% HashfunctionHeuristic(SHA-256)::0000000002010000002876657269666963617475
% 6d2e63727970746f2e4861736866756e6374696f6e486575726973746963010000000753
% 48412d323536
% \end{lstlisting}
% We can reformat the payload to emphasize its components:
% \begin{align*}
%   &\hex{00\cutt00\cut00\cut00\cut02}\\
%   &\quad\hex{01\cutt00\cut00\cut00\cut28}\\
%   &\quad\quad\hex{8766572696669636174756d2e63727970746f2e4861736866756e6374696}\\
%   &\quad\quad\hex{f6e486575726973746963}\\
%   &\quad\hex{01\cutt00\cut00\cut00\cut07\cutt5348412d323536}
% \end{align*}
% \end{example}




\section{Protocol Info Files}\label{sect:infofiles}

The protocol info file contains all the public parameters agreed on by
the operators before the key generation phase of the mix-net is
executed, and some of these parameters must be extracted to verify the
correctness of an execution.

\subsection{XML Grammar}\label{sect:xmlgrammar}

A protocol info file uses a simple XML format and contains a single
\xml{<protocol></protocol>} block. The preamble of this block contains
a number of global parameters, e.g., the number $k$ of parties
executing the protocol is given by a \xml{<nopart>$k$</nopart>} block,
and the group over which the protocol is executed is defined by a
\xml{<pgroup>\hex{123ABC}</pgroup>} block, where \hex{123ABC} is a
hexadecimal encoding of a byte tree representing the group. After the
global parameters follows one \xml{<party></party>} block for each
party that takes part in the protocol, and each such block contains
all the public information about that party, i.e., the name of a party
is given by a \xml{<name></name>} block. The contents of the
\xml{<party></party>} blocks are important during the execution of the
protocol, but they are not used to verify the correctness of an
execution and can be ignored.

A parser of protocol info files must be implemented. If
\code{protocolInfo.xml} is a protocol info file, then we denote by
$p=\ProtocolInfo(\code{protocolInfo.xml})$ an object such that
$\field{p}{b}$ is the data $d$ stored in a block \xml{<$b$>$d$</$b$>}
in the preamble of the protocol info file, i.e., preceding any
\xml{<party></party>} block. We stress that the data is stored as
ASCII encoded strings.

\listref{infofileskel} gives a skeleton example of a protocol info
file, where $\hex{"123ABC"}$ is used as a placeholder for some
hexadecimal rendition of an arithmetic or cryptographic
object. \listref{exampleinfofile} in \app{exampleinfofile} contains a
complete example of a protocol info file.

\begin{lstlisting}[frame=single,language=xml,
basicstyle=\tt\xmlfontsize,captionpos=b,
float=h,
caption={Skeleton of a protocol info file. There are no nested blocks within a \xml{<party></party>} block.},label=list:infofileskel]
<protocol>

   <name>Swedish Election</name>
   <nopart>3</nopart>
   <pgroup>123ABC</pgroup>
   ...

   <party>
      <name>Party1</name>
      <pubkey>123ABC</pubkey>
      ...
   </party>
   ...
</protocol>
\end{lstlisting}

\listref{protocolinfoschema} in \app{protocolinfoschema} contains the
formal XML schema for protocol info files, but this is not really
needed to implement a parser. In fact, to keep things simple, we do
not use any attributes of XML tags, i.e., all values are stored as the
data between an opening tag and a closing tag. The \code{vmni} command
can be used to output the schema in electronic form~\cite{vmnum}.

\subsection{Extracted Values}\label{sect:extracted}

To interpret an ASCII string $s$ as an integer we simply write
$\inte{s}$, e.g., $\inte{\code{"123"}}=123$. We let
$p=\code{ProtocolInfo(protocolInfo.xml)}$ and define the values we
later use in \sect{basicverify} and \sect{verify}.

\begin{itemize}

% \item $\mathrm{SID}=\field{p}{\code{sid}}$ is the globally unique session
%   identifier.

\item $\Mk=\inte{\field{p}{\code{nopart}}}$ specifies the number of parties.

\item $\thres=\inte{\field{p}{\code{thres}}}$ specifies the number of
  mix-servers that take part in the shuffling, i.e., this is the
  threshold number of mix-servers that must be corrupted to break the
  privacy of the senders.

\item $\PNmax=\inte{\field{p}{\code{maxciph}}}$ specifies the
  maximal number of ciphertexts (the actual number of ciphertexts is
  later denoted by $\PN$). This is zero if no pre-computation for a
  maximal number of ciphertexts took place before the mix-net was
  executed.

\item $\secpe=\inte{\field{p}{\code{vbitlenro}}}$ specifies the
  number of bits in each component of random vectors used for batching
  in proofs of shuffles and proofs of correct decryption.

\item $\secpr=\inte{\field{p}{\code{statdist}}}$ specifies the
  acceptable statistical error when sampling random values. The
  precise meaning of this parameter is hard to describe. Loosely,
  randomly chosen elements in the protocol are chosen with a
  distribution at distance at most roughly $2^{-\secpr}$ from uniform.

\item $\secpv=\inte{\field{p}{\code{cbitlenro}}}$ specifies the
  number of bits used in the challenge of the verifier in
  zero-knowledge proofs, i.e., in our Fiat-Shamir proofs it is the bit
  length of outputs from the random oracle $\RO_v$ defined in
  \sect{randomoracle}.

% \item $\hash=\unmarshal{\field{p}{\code{rohash}}}$ specifies the
%   hash function used to implement the random oracles.

% \item $\prg=\unmarshal{\field{p}{\code{prg}}}$ specifies the
%   pseudo-random generator used to expand challenges into arrays.

\item $\hash=\Hashfunction(\field{p}{\code{rohash}})$ specifies the
  hash function used to implement the random oracles.

\item $\prg=\PRG\big(\Hashfunction(\field{p}{\code{prg}})\big)$ specifies the
  pseudo-random generator used to expand challenges into arrays.

\item $\Gq=\unmarshal{\field{p}{\code{pgroup}}}$ specifies the
  underlying group.

\end{itemize}




\section{Verifying Fiat-Shamir Proofs}\label{sect:basicverify}

We use three different Fiat-Shamir proofs: a proof of a shuffle of
Pedersen commitments, a commitment-consistent proof of a shuffle of
ciphertexts, and a proof of correct decryption factors. We simply
write $\bt{a}$ for the byte tree representation of an object $a$.

\subsection{Random Oracles}

Throughout this section we use the following two random oracles
constructed from the hash function $\hash$, the minimum number
$\secps=\seedlen{\prg}$ of seed bits required by the pseudo-random
generator $\prg$, and the auxiliary security parameter $\secpv$.
\begin{itemize}

\item $\RO_s=\RandomOracle(\hash,\secps)$ is the random oracle used
  to generate seeds to $\prg$.

\item $\RO_v=\RandomOracle(\hash,\secpv)$ is the random oracle used
  to generate challenges.

\end{itemize}


\subsection{Independent Generators}\label{sect:independent}

The protocols in \sect{pos} and \sect{ccpos} also require
``independent'' generators and these generators must be derived using
the random oracles. To do that a seed
\begin{align*}
  s=\RO_s(\rho\mid\leaf{\code{"generators"}})
\end{align*}
is computed by hashing a prefix $\rho$ derived from the protocol file
and a string specifying the intended use of the ``independent''
generators. Then the generators are defined by
\begin{align*}
  \h=(\hi{0},\ldots,\hi{\PNmax-1})=\random{\Gq}{\PNmax,\prg(s),\secpr}\enspace,
\end{align*}
which is defined in \sect{arithmetic}. The prefix $\rho$ is computed
in \step{roprefix} of the main verification routine in
\sect{verifyalg} and given as input to \algo{pos}, \algo{ccpos}, and
\algo{decrypt} below. It is essentially a hash digest of the contents
of the protocol info file. In particular this means that the
``independent'' generators for different underlying groups are
``independently'' generated, since the description of the underlying
group is found in the protocol info file.


\subsection{Proof of a Shuffle of Commitments}\label{sect:pos}

In the \veri mix-net the mix-servers commit themselves in a
pre-computation phase to permutations that are later used during the
mixing phase. A proof of a shuffle of commitments allows a mix-server
to show that it did so correctly and that it knows how to open its
commitment. Below we only describe the computations performed by the
verifier for a specific application of the Fiat-Shamir heuristic. For
a detailed description of the complete protocol including the
computations performed by the prover we refer the reader to
\app{proofsofshuffles} and Terelius and Wikstr{\"o}m~\cite{TW10}.

\begin{nicebox}
  \begin{algorithm}[Verifier of Proof of a Shuffle of Commitments]$\quad$\label{algo:pos}\\

    \vspace{-0.3cm}
    \hspace{-0.3cm}
    \begin{tabular}{cl}
      \textbf{Input} & \textbf{Description}\\
      $\rho$ & Prefix to random oracles.\\
      $\PNmax$ & Size of the arrays.\\
      $\secpe$ & Number of bits in each component of random
      vectors used for batching.\\
      $\secpr$ & Acceptable ``statistical error'' when deriving independent generators.\\
      $\secpv$ & Number of bits in challenges.\\
      $\prg$ & Pseudo-random generator $\prg$ used to derive random
      vectors for batching.\\
      $\Gq$ & Group of prime order with standard generator $g$.\\ 
      $u$ & Array $\permcommj{}=(\permcommjip{0},\ldots,\permcommjip{\PNmax-1})$ of Pedersen commitments in
      $\Gq$.\\
      $\proofC{}$ & Commitment of the Fiat-Shamir proof.\\
      $\proofR{}$ & Reply of the Fiat-Shamir proof.
    \end{tabular}

    \vspace{0.2cm}
    \noindent
    \textbf{Program}

    \vspace{-0.3cm}
  \begin{enumerate}

  \item 
    \begin{enumerate}
    \item Interpret $\proofC{}$ as
      $\node{\bt{B},\bt{A'},\bt{B'},\bt{C'},\bt{D'}}$, where
      $A',C',D'\in\Gq$, and $B=(B_0,\ldots,B_{\PNmax-1})$ and
      $B'=(B_0',\ldots,B_{\PNmax-1}')$ are arrays in $\Gq$.

  \item Interpret $\proofR{}$ as
    $\node{\bt{k_A},\bt{k_B},\bt{k_C},\bt{k_D},\bt{k_E}}$, where
    $k_A,k_C,k_D\in\zedq$, and $k_B=(k_{B,0},\ldots,k_{B,\PNmax-1})$ and
    $k_E=(k_{E,0},\ldots,k_{E,\PNmax-1})$ are arrays in $\zedq$.

    \end{enumerate}
    Reject if this fails.

  \item Compute a seed
    $s=\RO_s\big(\rho\mid\node{\bt{g},\bt{h},\bt{u}}\big)$.

  \item Set $(t_0,\ldots,t_{\PNmax-1})=\prg(s)$, where
    $t_i\in\{0,1\}^{8\lceil\secpe/8\rceil}$ is interpreted as a
    non-negative integer $0\leq t_i<2^{8\lceil\secpe/8\rceil}$, set $e_i=t_i\bmod 2^{\secpe}$ and compute
    \begin{align*}
      A=\prod\nolimits_{i=0}^{\PNmax-1}\permcommjip{i}^{e_i}\enspace .
    \end{align*}

  \item Compute a challenge
    $v=\RO_v\big(\rho\mid\node{\leaf{s},\proofC{}}\big)$ interpreted as a
    non-negative integer $0\leq v<2^{\secpv}$.

  \item Compute
    \begin{align*}
      C=\frac{\prod\nolimits_{i=0}^{\PNmax-1}\permcommjip{i}}{\prod\nolimits_{i=0}^{\PNmax-1}\hi{i}}\quad\text{and}\quad D=\frac{B_{\PNmax-1}}{\hi{0}^{\prod_{i=0}^{\PNmax-1}e_i}}\enspace,
    \end{align*}
    set $B_{-1}=\hi{0}$, and accept if and only if:
    \begin{align*}
      A^vA'&=g^{k_A}\prod\nolimits_{i=0}^{\PNmax-1}\hi{i}^{k_{E,i}}
      &C^vC'&=g^{k_C}\\
      B_i^vB_i'&=g^{k_{B,i}}B_{i-1}^{k_{E,i}}\enspace\text{for $i=0,\ldots,\PNmax-1$}
      &D^vD'&=g^{k_D}
    \end{align*}
  \end{enumerate}
  
\end{algorithm}
\end{nicebox}

\subsection{Commitment-Consistent Proof of a Shuffle of Ciphertexts}\label{sect:ccpos}

During the mixing in \veri each mix-server re-encrypts the ciphertexts
in its input, permutes the resulting ciphertexts using the permutation
it is committed to, and then outputs the result. Then it uses a
commitment-consistent proof of a shuffle to show that it did so
correctly. We only describe a specific implementation of the verifier
using the Fiat-Shamir heuristic. For a detailed description of the
complete protocol including the computations performed by the prover
we refer the reader to \app{proofsofshuffles} and
Wikstr{\"o}m~\cite{wik09b}.

\begin{nicebox}
  \begin{algorithm}[Verifier of Commitment-Consistent Proof of a Shuffle]$\quad$\label{algo:ccpos}\\

    \vspace{-0.3cm}
    \hspace{-0.3cm}
    \begin{tabular}{cl}
      \textbf{Input} & \textbf{Description}\\
      $\rho$ & Prefix to random oracles.\\
      $\PN$ & Size of the arrays.\\
      $\secpe$ & Number of bits in each component of random
      vectors used for batching.\\
      $\secpr$ & Acceptable ``statistical error'' when deriving independent generators.\\      
      $\secpv$ & Number of bits in challenges.\\
      $\prg$ & Pseudo-random generator $\prg$ used to derive random
      vectors for batching.\\
      $\Gq$ & Group of prime order.\\ 
      $u$ & Shrunk array $\permcommj{}=(\permcommjip{0},\ldots,\permcommjip{\PN})$ of Pedersen commitments in
      $\Gq$.\\
      $\RSpace$ & Randomizer group.\\
      $\CSpace$ & Ciphertext group.\\
      $\pk$ & \elgamal public key.\\ 
      $\ciphj{}$ & Array $\ciphj{}=(\ciphjip{0},\ldots,\ciphjip{\PN-1})$ of input ciphertexts in $\CSpace$.\\
      $\ciphj{}'$ & Array $\ciphj{}'=(\ciphjip{0}',\ldots,\ciphjip{\PN-1}')$ of output ciphertexts in $\CSpace$.\\
      $\proofC{}$ & Commitment of the Fiat-Shamir proof.\\
      $\proofR{}$ & Reply of the Fiat-Shamir proof.
    \end{tabular}

    \vspace{0.2cm}
    \noindent
    \textbf{Program}

    \vspace{-0.3cm}
  \begin{enumerate}

  \item 
    \begin{enumerate}

    \item Interpret $\proofC{}$ as $\node{\bt{A'},\bt{B'}}$, where
      $A'\in\Gq$ and $B'\in\CSpace$.

    \item Interpret $\proofR{}$ as
      $\node{\bt{k_A},\bt{k_B},\bt{k_E}}$, where $k_A\in\zedq$,
      $k_B\in\RSpace$, and $k_E$ is an array of $\PN$ elements in
      $\zedq$.

    \end{enumerate}
    Reject if this fails.

  \item Compute a seed
    $s=\RO_s\big(\rho\mid\node{\bt{g},\bt{h},\bt{u},\bt{\pk},\bt{\ciphj{}},\bt{\ciphj{}'}}\big)$.

  \item Set $(t_0,\ldots,t_{\PN-1})=\prg(s)$, where
    $t_i\in\{0,1\}^{8\lceil\secpe/8\rceil}$ is interpreted as a
    non-negative integer $0\leq t_i<2^{8\lceil\secpe/8\rceil}$, set $e_i=t_i\bmod 2^{\secpe}$ and compute
    \begin{align*}
      A=\prod_{i=0}^{\PN-1}\permcommjip{i}^{e_i}\enspace.
    \end{align*}

  \item Compute a challenge
    $v=\RO_v\big(\rho\mid\node{\leaf{s},\proofC{}}\big)$ interpreted as a
    non-negative integer $0\leq v<2^{\secpv}$.

  \item Compute $B=\prod\nolimits_{i=0}^{N-1}\ciphjip{i}^{e_i}$ and accept if
    and only if:
    \begin{align*}
      A^vA'&=g^{k_A}\prod\nolimits_{i=0}^{N-1}\hi{i}^{k_{E,i}}
      &B^vB'&=\Enc_{\pk}(1,-k_{B})\prod\nolimits_{i=0}^{N-1}(\ciphjip{i}')^{k_{E,i}}\\
    \end{align*}

  \end{enumerate}
\end{algorithm}
\end{nicebox}


\subsection{Proof of Correct Decryption Factors}

At the end of the mixing the parties jointly decrypt the re-encrypted
and permuted list of ciphertexts. To prove that they did so correctly
they use a proof of correct decryption factors. This is a standard
protocol using batching for improved efficiency. The general technique
originates in Bellare et al.~\cite{BeGaRa}.

\begin{nicebox}
\begin{algorithm}[Verifier of Decryption Factors]\label{algo:decrypt}$\quad$\\

    \vspace{-0.3cm}
    \hspace{-0.3cm}
    \begin{tabular}{cl}
      \textbf{Input} & \textbf{Description}\\
      $\rho$ & Prefix to random oracles.\\
      $\PN$ & Size of the arrays.\\
      $\secpe$ & Number of bits in each component of random
      vectors used for batching.\\
      $\secpr$ & Acceptable ``statistical error'' when deriving independent generators.\\      
      $\secpv$ & Number of bits in challenges.\\
      $\prg$ & Pseudo-random generator $\prg$ used to derive random
      vectors for batching.\\
      $\Gq$ & Group of prime order.\\ 
      $\y$ & Partial public key.\\ 
      $\CSpace$ & Ciphertext group.\\
      $\MSpace$ & Plaintext group.\\
      $\ciphj{}$ & Array $\ciphj{}=(\ciphjip{0},\ldots,\ciphjip{\PN-1})$ of input ciphertexts in $\CSpace$.\\
      $\decrfact$ & Array $\decrfact=(\decrfactjip{0},\ldots,\decrfactjip{\PN-1})$ of decryption factors in $\Gq$.\\
      $\proofC{}$ & Commitment of the Fiat-Shamir proof.\\
      $\proofR{}$ & Reply of the Fiat-Shamir proof.
    \end{tabular}

    \vspace{0.2cm}
    \noindent
    \textbf{Program}

    \vspace{-0.3cm}
    \begin{enumerate}

    \item 
      \begin{enumerate}
      \item Interpret $\proofC{}$ as $\node{\bt{\y'},\bt{B''}}$, where
        $\y'\in\Gq$ and $B''$ is an array containing a single element $B'$
        in $\MSpace$.

      \item Interpret $\proofR{}$ as $\bt{k_{\x}}$, where
        $k_{\x}\in\zedq$.

      \end{enumerate}
      Reject if this fails.

    \item Compute a seed
      $s=\RO_s\big(\rho\mid\node{\node{\bt{g},\bt{\ciph}},\node{\bt{\y},\bt{\decrfact}}}\big)$.

    \item Set $(t_0,\ldots,t_{\PN-1})=\prg(s)$, where
      $t_i\in\{0,1\}^{8\lceil\secpe/8\rceil}$ is interpreted as a
      non-negative integer $0\leq t_i<2^{8\lceil\secpe/8\rceil}$, set $e_i=t_i\bmod 2^{\secpe}$ and compute
      \begin{align*}
        A=\prod\nolimits_{i=0}^{\PN-1}\ciphjip{i}^{e_i}\quad\text{and}\quad
        B=\prod\nolimits_{i=0}^{\PN-1}\decrfact^{e_i}\enspace .
      \end{align*}

    \item Compute a challenge
      $v=\RO_v\big(\rho\mid\node{\leaf{s},\proofC{}}\big)$ interpreted as a
      non-negative integer $0\leq v<2^{\secpv}$.

    \item Accept if and only if $\y^v\y'=\g^{k_{\x}}$ and
      $B^vB'=\PDec_{k_{\x}}(A)$.

    \end{enumerate}

\end{algorithm}
\end{nicebox}


\section{Verification}\label{sect:verify}

The verification algorithm must verify that the input ciphertexts were
repeatedly re-rerandomized by the mix-servers and then jointly
decrypted with a secret key corresponding to the public key used by
senders to encrypt their messages. Furthermore, the parameters of the
execution must match the parameters in the protocol info file.

The verification algorithm must take two parameters.
\begin{itemize}

\item \code{protocolInfo.xml} -- Protocol info file containing the
  public parameters. \sect{infofiles} describes the format of this
  file and \sect{extracted} introduces notation for the values we
  need.
    
\item \code{roProof} -- Proof directory containing the intermediate
  results and the Fiat-Shamir proofs relating the intermediate
  results. The contents of this directory is described below.
 
\end{itemize}

\subsection{Contents of the Proof Directory}\label{sect:inputs}

The proof directory contains not only the Fiat-Shamir proofs, but also
the intermediate results. In this section we describe the formats of
these files and introduce notation for their contents. Here
$\param{l}$ denotes a integer parameter $0\leq l\leq \Mk$ encoded
using two decimal digits representing the index of a mix-server, but a
file with suffix $l$ may not originate from the $l$th mix-server if it
is corrupted and all types of files do not appear with all
suffixes. For easy reference we tabulate the notation introduced below
and from which file the contents is derived in \tabl{corr}.
\begin{enumerate}

\item\label{poin:fullpk}\btfilei{FullPublicKey}{} -- Full public key
  used to form input ciphertexts.

\item\label{poin:pk}\btfilei{PublicKey}{$\param{l}$} -- Partial public
  key of the $l$th mix-server. The required format of this file is a
  byte tree $\bt{\yj{l}}$, where $\yj{l}\in\Gq$.

\item\label{poin:sk}\btfilei{SecretKey}{$\param{l}$} -- Secret key
  file of the $l$th party. This is only created if the $l$th
  mix-server is identified as a cheater. In this case its secret key
  is recovered and its part in the joint decryption is computed
  locally by the other mix-servers. The required format of this file
  is a byte tree $\bt{\xj{l}}$, where $\xj{l}\in\zedq$.

\item\label{poin:ciphs}\btfilei{CiphertextList}{$\param{l}$} -- The
  $l$th intermediate list of ciphertexts, i.e., normally the output of
  the $l$th mix-server. This file should contain a byte tree
  $\bt{\List{l}}$, where $\List{l}$ is an array of $\PN$ elements in
  $\CSpace$ and $\PN\leq\PNmax$ is the number of elements in the list
  $\List{0}$ of input ciphertexts.

\item\label{poin:permcom}\btfilei{PermutationCommitment}{$\param{l}$}
  -- Commitment to a permutation. This file should contain a byte tree
  $\bt{\permcommj{l}}$, where $\permcommj{l}$ is an array of $\PNmax$
  elements in $\Gq$.

\item\label{poin:poscom}\btfilei{PoSCommitment}{$\param{l}$} --
  ``Proof commitment'' of the proof of a shuffle of commitments. The
  required format of the byte tree $\posproofC{l}$ in this file is
  specified in \algo{pos}.

\item\label{poin:posrep}\btfilei{PoSReply}{$\param{l}$} -- ``Proof
  reply'' of the proof of a shuffle of commitments. The required
  format of the byte tree $\posproofR{l}$ in this file is specified in
  \algo{pos}.

\item\label{poin:keep}\btfilei{KeepList}{$\param{l}$} -- Keep-list
  used to shrink a permutation-commitment if pre-computation is used
  before the mix-net is executed. The file should contain a byte tree
  $\bt{\keeplistj{l}}$, where $\keeplistj{l}$ should be an array of
  $\PNmax$ booleans, of which exactly $N$ are true, indicating which
  components to keep.

\item\label{poin:ccposcom}\btfilei{CCPoSCommitment}{$\param{l}$} --
  ``Proof commitment'' of the commitment-consistent proof of a
  shuffle. The required format of the byte tree $\ccposproofC{l}$ in
  this file is specified in \algo{ccpos}.

\item\label{poin:ccposrep}\btfilei{CCPoSReply}{$\param{l}$} -- ``Proof
  reply'' of the commitment-consistent proof of a shuffle. The
  required format of the byte tree $\ccposproofR{l}$ in this file is
  specified in \algo{ccpos}.

\item\label{poin:decr}\btfilei{DecryptionFactors}{$\param{l}$} --
  Decryption factors of the $l$th mix-server used to jointly decrypt
  the shuffled ciphertexts. This file should contain a byte tree
  $\node{\bt{\yj{l}},\bt{\decrfactj{l}}}$, where $\decrfactj{l}$ is an
  array of $\PN$ elements in $\Gq$.

\item\label{poin:decrcom}\btfilei{DecrFactCommitment}{$\param{l}$} --
  ``Proof commitment'' of the proof of correctness of the decryption
  factors. The required format of the byte tree $\dproofC{l}$ of this
  file is specified in \algo{decrypt}.

\item\label{poin:decrrep}\btfilei{DecrFactReply}{$\param{l}$} --
  ``Proof reply'' of the proof of correctness of the decryption
  factors. The required format of the byte tree $\dproofC{l}$ of this
  file is specified in \algo{decrypt}.

\item\label{poin:plain}\btfilei{PlaintextElements}{} -- The output
  plaintext elements that has not been decoded in any way. This file
  should contain a byte tree $\bt{m}$, where $m$ is an array of $\PN$
  elements in $\MSpace$.

\end{enumerate}

\begin{table}[h]
  \centering
\begin{tabular}{c|c|l}
  Notation & Point & File \\
  \hline
  $(\g,\y)$ & \ref{poin:fullpk} & \btfilei{FullPublicKey}{}\\
  $\yj{l}$ & \ref{poin:pk} & \btfilei{PublicKey}{$\param{l}$}\\
  $\xj{l}$ & \ref{poin:sk} & \btfilei{SecretKey}{$\param{l}$}\\
  $\List{l}$ & \ref{poin:ciphs} & \btfilei{CiphertextList}{$\param{l}$}\\
  $\permcommj{l}$ & \ref{poin:permcom} & \btfilei{PermutationCommitment}{$\param{l}$}\\
  $\posproofC{l}$ & \ref{poin:poscom} & \btfilei{PoSCommitment}{$\param{l}$}\\
  $\posproofR{l}$ & \ref{poin:posrep} & \btfilei{PoSReply}{$\param{l}$}\\
  $\keeplistj{l}$ & \ref{poin:keep} & \btfilei{KeepList}{$\param{l}$}\\
  $\ccposproofC{l}$ & \ref{poin:ccposcom} & \btfilei{CCPoSCommitment}{$\param{l}$}\\
  $\ccposproofR{l}$ & \ref{poin:ccposrep} & \btfilei{CCPoSReply}{$\param{l}$}\\
  $\decrfactj{l}$ & \ref{poin:decr} & \btfilei{DecryptionFactors}{$\param{l}$}\\
  $\dproofC{l}$ & \ref{poin:decrcom} & \btfilei{DecrFactCommitment}{$\param{l}$}\\
  $\dproofC{l}$ & \ref{poin:decrrep} & \btfilei{DecrFactReply}{$\param{l}$}\\
  $\mess$ & \ref{poin:plain} & \btfilei{PlaintextElements}{}
\end{tabular}
\caption{Correspondence between notation and files containing the denoted objects.}\label{tabl:corr}
\end{table}


\subsection{Core Verification Algorithm}\label{sect:verifyalg}

We are finally ready to describe the core verification algorithm. We
stress that this algorithm only verifies that a protocol info file and
the contents of the proof directory are correct. In an application the
public key actually given to senders, the input ciphertexts, and the
output plaintexts may be represented using an application-dependent
format and it must be verified that these correspond to their
counterparts in the proof directory. We discuss this in
\sect{fullverify}.

\begin{enumerate}

\item\textbf{Verification of Protocol Info File.} The soundness of the
  parameters of the execution must be verified manually by a
  cryptographer. If any parameter is not chosen with an acceptable
  security level, then \reject. Verify that the XML of the protocol
  info file is compliant with the XML schema of protocol info files
  and \reject otherwise.

\item\textbf{Public Parameters.} Read the public parameters from the
  protocol info file as described in \sect{infofiles}. If this fails,
  then \reject. This defines
  \begin{align*}
    \Mk,\PNmax,\secpr,\secpv,\secpe,\prg,\Gq,\hash\enspace.
  \end{align*}

\item\label{step:roprefix}\textbf{Prefix to Random Oracles.} To
  differentiate executions on different public parameters, we let
  $\rho=\leaf{\hash(f)}$, where $f$ is the array of bytes contained in
  the protocol info file, and use this as a prefix to our random
  oracles.

\item\textbf{Joint Public Key.} Attempt to read the joint public key
  $(\g,\y)\in\Gq\times\Gq$ from file as described in \poin{fullpk}. If
  this fails, then \reject.

\item\textbf{Individual Public Keys.} Read public keys
  $\yj{1},\ldots,\yj{\Mk}$ as described in \poin{pk}. If this fails,
  then \reject. Then test if $\y=\prod_{l=1}^{\Mk}\yj{l}$, i.e., check
  that the keys $\yj{1},\ldots,\yj{\Mk}$ of the mix-servers are
  consistent with the public key $\y$. If not, then \reject.

\item\textbf{Array of Input Ciphertexts.} Read the array $\List{0}$ of
  $\PN\leq\PNmax$ ciphertexts as described in \poin{ciphs} for some
  $\PN$. If this fails, then \reject.

\item\textbf{Proofs of Shuffles.} For $l=1,\ldots,\thres$ do:
  \begin{enumerate}

  \item\textbf{Verify Proof of Shuffle of Commitments.}  Read a
    permutation commitment $\permcommj{l}$, a proof commitment
    $\posproofC{l}$, and proof reply $\posproofR{l}$ as described in
    \poin{permcom}, \poin{poscom}, and \poin{posrep},
    respectively. Then execute
    \algo{pos} on input\\
    $(\rho,\PNmax,\secpe,\secpr,\secpv,\prg,\Gq,\permcommj{l},\posproofC{l},\posproofR{l})$. If
    reading fails or if the algorithm rejects, then set
    $\permcommj{l}=h$.

  \item\textbf{Shrink Permutation Commitment.} 
    \begin{enumerate}

    \item Read the keep-list $\keeplistj{l}$ as described in
      \poin{keep}. If this fails, then let $\keeplistj{l}$ be the
      array of $\PNmax$ booleans of which the first $\PN$ are true and
      the rest false.

    \item Set
      $\spermcommj{l}=(\permcommji{l}{i})_{\keeplistji{l}{i}=\True}$
      be the sub-array indicated by $\keeplistj{l}$.

    \end{enumerate}

  \item\textbf{Array of Ciphertexts.} Read the array $\List{l}$ of
    $\PN$ ciphertexts in $\CSpace$ as described in \poin{ciphs}. If
    this fails, then \reject.

  \item\textbf{Verify Commitment-Consistent Proof of Shuffle.} Read
    proof commitment $\ccposproofC{l}$ and proof reply
    $\ccposproofR{l}$ as described in \poin{ccposcom} and
    \poin{ccposrep}, respectively. Then execute \algo{ccpos} on input
    \begin{align*}
      (\rho,\PNmax,\PN,\secpe,\secpr,\secpv,\prg,\Gq,\permcommj{l},\RSpace,\CSpace,\List{l-1},\List{l},\pk,\ccposproofC{l},\ccposproofR{l})\enspace.
    \end{align*}
    If reading fails or if the algorithm rejects, then verify that
    $\List{l}=\List{l-1}$. If this is not the case, then \reject.

  \end{enumerate}

\item\textbf{Proofs of Decryption.} For $l=1,\ldots,\Mk$ do:
  \begin{enumerate}

  \item If $\xj{l}$ can be read as described in \poin{sk} such that
    $\yj{l}=\g^{\xj{l}}$, then set
    $\decrfactj{l}=\PDec_{\xj{l}}(\List{l-1})$.

  \item Otherwise, read $\decrfactj{l}$, $\proofC{l}$, and
    $\proofR{l}$ of the $l$th mix-server as described in \poin{decr},
    \poin{decrcom}, and \poin{decrrep}, respectively. Then execute
    \algo{decrypt} on input
    \begin{align*}
      (\rho,\PN,\secpe,\secpr,\secpv,\prg,\Gq,\permcomm,\CSpace,\MSpace,\y,\List{\thres},\decrfactj{l},\proofC{l},\proofR{l})\enspace.
    \end{align*}
    If it rejects, then \reject.

  \end{enumerate}

\item\textbf{Verify Output.} Attempt to read the plaintext elements
  $\mess$ as described in \poin{plain}. Then verify that
  $\mess=\TDec\big(\List{\thres},\prod_{l=1}^{\Mk}\decrfactj{l}\big)$.

\end{enumerate}


\paragraph{Standard Command-Line Interface of Core Verification Algorithm.}

To simplify the implementation of scripts that execute multiple
verifiers that are implemented by independent parties, we require that
every verifier can be used as follows, where \code{vmnv} denotes the
verification algorithm command, \code{protInfo.xml} is the protocol
info file, and \code{roProof} is the proof directory.

\vspace{0.3cm}
\begin{lstlisting}[frame=single,language=xml,
basicstyle=\tt,showstringspaces=false]
vmnv protInfo.xml roProof
\end{lstlisting}

\vspace{0.2cm}
\noindent
The command should by default give no output and its exit status
should be zero if the proof is accepted and non-zero otherwise. It
should also have an option \code{-v} that activates verbose output,
but it is up to the implementer to decide on the format used in the
output. It is acceptable to assume that the option appears before any
of the parameters.

\subsection{Verification Algorithm}\label{sect:fullverify}

The formats used to represent the public key handed to the senders and
the list of ciphertexts received from senders and how plaintext group
elements are decoded into messages is application dependent. Thus, to
verify the overall correctness, it must be verified that:
\begin{enumerate}

\item The public key actually used by senders is $(\g,\y)$.

\item The list of ciphertexts actually submitted by senders is
  $\List{0}$.

\item The output plaintexts are correctly decoded from the group
  elements in $\mess$.

\end{enumerate}
This falls outside the scope of this document, since in general we can
not anticipate the representations used by application developer or
how the plaintext group elements are interpreted.

However, one natural approach is to use an existing or custom mix-net
interface of \veri. In the user manual of the \veri
mix-net~\cite{vmnum} we describe how this can be done.


\paragraph{Standard Command-Line Interface of Verification Algorithm.}

Regardless of the approach used, we require that the verifier
implements a standard command-line interface to allow easy
scripting. Let \code{vmnv} denote the verification algorithm command,
let \code{protInfo.xml} be the protocol info file, let \code{roProof}
be the proof directory, let \code{publicKey} contain the public key,
let \code{ciphertexts} contain the input ciphertexts, and let
\code{plaintexts} contain the decoded plaintexts. Then we require that
the following runs the verification algorithm.

\vspace{0.3cm}
\begin{lstlisting}[frame=single,language=xml,
basicstyle=\tt,showstringspaces=false]
vmnv protInfo.xml roProof pkey ciphertexts plaintexts
\end{lstlisting}

\vspace{0.2cm}
\noindent
The \code{-v} option must activate verbose output in the same way as
for the core verification command. We stress that every verifier
should also implement the core verifier command-line interface.


\section{Acknowledgments}

Olivier Pereira gave valuable feedback on an initial version of this
document. The suggestions of Tomer Hasid, Shahram Khazaei, Gunnar
Kreitz, and Amnon Ta-Shma have greatly improved the presentation.


\bibliographystyle{abbrv}
\bibliography{vtm}


\appendix

\section{Test Vectors for the Cryptographic Library}

\begin{align*}
&\PRG(\Hashfunction(\leaf{\code{"SHA-256}"}))\\
&\textrm{Seed (32 bytes):}\\
&\hex{000102030405060708090a0b0c0d0e0f101112131415161718191a1b1c1d1e1f}\\
&\textrm{Expansion (128 bytes):}\\
&\hex{70f4003d52b6eb03da852e93256b5986b5d4883098bb7973bc5318cc66637a84}\\
&\hex{04a6950a06d3e3308ad7d3606ef810eb124e3943404ca746a12c51c7bf776839}\\
&\hex{0f8d842ac9cb62349779a7537a78327d545aaeb33b2d42c7d1dc3680a4b23628}\\
&\hex{627e9db8ad47bfe76dbe653d03d2c0a35999ed28a5023924150d72508668d244}
\end{align*}
\begin{align*}
&\PRG(\Hashfunction(\leaf{\code{"SHA-384}"}))\\
&\textrm{Seed (48 bytes):}\\
&\hex{000102030405060708090a0b0c0d0e0f101112131415161718191a1b1c1d1e1f}\\
&\hex{202122232425262728292a2b2c2d2e2f}\\
&\textrm{Expansion (128 bytes):}\\
&\hex{e45ac6c0cafff343b268d4cbd773328413672a764df99ab823b53074d94152bd}\\
&\hex{27fc38bcffdb7c1dc1b6a3656b2d4819352c482da40aad3b37f333c7afa81a92}\\
&\hex{b7b54551f3009efa4bdb8937492c5afca1b141c99159b4f0f819977a4e10eb51}\\
&\hex{61edd4b1734717de4106f9c184a17a9b5ee61a4399dd755f322f5d707a581cc1}
\end{align*}
\begin{align*}
&\PRG(\Hashfunction(\leaf{\code{"SHA-512}"}))\\
&\textrm{Seed (64 bytes):}\\
&\hex{000102030405060708090a0b0c0d0e0f101112131415161718191a1b1c1d1e1f}\\
&\hex{202122232425262728292a2b2c2d2e2f303132333435363738393a3b3c3d3e3f}\\
&\textrm{Expansion (128 bytes):}\\
&\hex{979043771043f4f8e0a2a19b1fbfbe5a8f076c2b5ac003e0b9619e0c45faf767}\\
&\hex{47295734980602ec1d8d3cd249c165b7db62c976cb9075e35d94197c0f06e1f3}\\
&\hex{97a45017c508401d375ad0fa856da3dfed20847716755c6b03163aec2d9f43eb}\\
&\hex{c2904f6e2cf60d3b7637f656145a2d32a6029fbda96361e1b8090c9712a48938}
\end{align*}
\begin{align*}
&\RandomOracle(\Hashfunction(\leaf{\code{"SHA-256}"}),65)\\
&\textrm{Input (32 bytes):}\\
&\hex{000102030405060708090a0b0c0d0e0f101112131415161718191a1b1c1d1e1f}\\
&\textrm{Output (9 bytes of which the last 65 bits may be non-zero):}\\
&\hex{001a8d6b6f65899ba5}
\end{align*}
\begin{align*}
&\RandomOracle(\Hashfunction(\leaf{\code{"SHA-256}"}),261)\\
&\textrm{Input (32 bytes):}\\
&\hex{000102030405060708090a0b0c0d0e0f101112131415161718191a1b1c1d1e1f}\\
&\textrm{Output (33 bytes of which the last 261 bits may be non-zero):}\\
&\hex{1c04f57d5f5856824bca3af0ca466e283593bfc556ae2e9f4829c7ba8eb76db8}\\
&\hex{78}
\end{align*}
\begin{align*}
&\RandomOracle(\Hashfunction(\leaf{\code{"SHA-384}"}),93)\\
&\textrm{Input (32 bytes):}\\
&\hex{000102030405060708090a0b0c0d0e0f101112131415161718191a1b1c1d1e1f}\\
&\textrm{Output (12 bytes of which the last 93 bits may be non-zero):}\\
&\hex{04713a5e22935833d436d1db}
\end{align*}
\begin{align*}
&\RandomOracle(\Hashfunction(\leaf{\code{"SHA-384}"}),411)\\
&\textrm{Input (32 bytes):}\\
&\hex{000102030405060708090a0b0c0d0e0f101112131415161718191a1b1c1d1e1f}\\
&\textrm{Output (52 bytes of which the last 411 bits may be non-zero):}\\
&\hex{00dc086c320e38b92722a9c0f87f2f5de81b976400e2441da542d1c3f3f391e4}\\
&\hex{1d6bcd8297c541c2431a7272491f496b622266aa}
\end{align*}
\begin{align*}
&\RandomOracle(\Hashfunction(\leaf{\code{"SHA-512}"}),111)\\
&\textrm{Input (32 bytes):}\\
&\hex{000102030405060708090a0b0c0d0e0f101112131415161718191a1b1c1d1e1f}\\
&\textrm{Output (14 bytes of which the last 111 bits may be non-zero):}\\
&\hex{28d742c34b97367eb968a3f28b6c}
\end{align*}
\begin{align*}
&\RandomOracle(\Hashfunction(\leaf{\code{"SHA-512}"}),579)\\
&\textrm{Input (32 bytes):}\\
&\hex{000102030405060708090a0b0c0d0e0f101112131415161718191a1b1c1d1e1f}\\
&\textrm{Output (73 bytes of which the last 579 bits may be non-zero):}\\
&\hex{00a6f79b8450fef79af71005c0b1028c9f025f322f1485c2b245f658fe641d47}\\
&\hex{dcbb4fe829e030b52e4a81ca35466ad1ca9be6feccb451e7289af318ddc9dae0}\\
&\hex{98a5475d6119ff6fe0}
\end{align*}


\section{Schema for Protocol Info Files}\label{sect:protocolinfoschema}

\lstset{
%frame=single,
language=xml,
keywordstyle=\tt\xmlfontsize,
float,
basicstyle=\tt\xmlfontsize,
captionpos=t,
label=list:protocolinfoschema
}

\lstinputlisting{protInfo.xsd}

\section{Example Protocol Info File}\label{sect:exampleinfofile}

\lstset{
%frame=single,
keywordstyle=\it\ttfamily\xmlfontsize,
language=xml,
float,
basicstyle=\tt\xmlfontsize,
captionpos=t,
label=list:exampleinfofile
}

\lstinputlisting{protInfo.tex}

\section{Protocols}\label{sect:proofsofshuffles}

\begin{nicebox}
\begin{protocol}[Proof of a Shuffle of Commitments]$\quad$\label{prot:matrix}\\
  \textbf{Common Input.} Generators $g,h_0,\ldots,h_{N-1}\in\Gq$
  and
  Pedersen commitments $u_0,\ldots,u_{N-1}\in\Gq$.\\
  \textbf{Private Input.} Exponents $r=(r_0,\ldots,r_{N-1})\in\zedq^N$
  and a permutation $\pi\in\permset{N}$ such that
  $u_i=g^{r_{\pi(i)}}h_{\pi(i)}$ for $i=0,\ldots,N-1$.
  \begin{enumerate}

  \item $\V$ chooses a seed $s\in\{0,1\}^{\secp}$ randomly, defines
    $e\in\INT{\secpe}^N$ as $e=\PRG(s)$, hands $s$ to $\P$ and
    computes,
    \begin{align*}
      A=\prod\nolimits_{i=0}^{N-1}u_i^{e_i}\enspace .
    \end{align*}

  \item $\P$ computes the following, where $e_i'=e_{\ipi(i)}$:
    \begin{enumerate}
    \item \emph{Bridging Commitments.} It chooses
      $b_0,\ldots,b_{N-1}\in\zedq$ randomly, sets $B_{-1}=h_0$, and forms
    \begin{align*}
      B_i=g^{b_i}B_{i-1}^{e_i'}\enspace
      \text{for $i=0,\ldots,N-1$}\enspace.
    \end{align*}

  \item \emph{Proof Commitments.} It chooses
    $\alpha,\beta_0,\ldots,\beta_{N-1},\gamma,\delta\in\zedq$ and
    $\epsilon_0,\ldots,\epsilon_{N-1}\in\INT{\secpe+\secpv+\secpr}$
    randomly, sets $B_{-1}=h_0$, and forms
    \begin{align*}
      A'&=g^{\alpha}\prod\nolimits_{i=0}^{N-1}h_i^{\epsilon_i}
      &C'&=g^\gamma\\
      B_i'&=g^{\beta_i}B_{i-1}^{\epsilon_i}\enspace\text{for $i=0,\ldots,N-1$}
      &D'&=g^\delta\enspace.
    \end{align*}
    \end{enumerate}
    Then it hands $(B,A',B',C',D')$ to $\V$.

  \item $\V$ chooses $v\in\INT{\secpv}$ randomly and hands $v$ to $\P$.

  \item $\P$ computes $a=\langle r,e'\rangle$,
    $c=\sum_{i=0}^{N-1}r_i$. Then it sets $d_0=b_0$ and computes
    $d_i=b_i+e_i'd_{i-1}$ for $i=1,\ldots,N-1$. Finally, it sets
    $d=d_{N-1}$ and computes
    \begin{align*}
      k_A&=va+\alpha
      &k_C&=vc+\gamma \\
      k_{B,i}&=vb_i+\beta_i\enspace\text{for $i=0,\ldots,N-1$}
      &k_D&=vd+\delta\\
      &&k_{E,i}&=ve_i'+\epsilon_i\enspace\text{for $i=0,\ldots,N-1$}\enspace.
    \end{align*}
    Then it hands $(k_A,k_B,k_C,k_D,k_E)$ to $\V$.

  \item $\V$ computes
    \begin{align*}
      C=\frac{\prod\nolimits_{i=0}^{N-1}u_i}{\prod\nolimits_{i=0}^{N-1}h_i}\quad\text{and}\quad D=\frac{B_{N-1}}{h_0^{\prod_{i=0}^{N-1}e_i}}\enspace,
    \end{align*}
    sets $B_{-1}=h_0$ and accepts if and only if
    \begin{align*}
      A^vA'&=g^{k_A}\prod\nolimits_{i=0}^{N-1}h_i^{k_{E,i}}
      &C^vC'&=g^{k_C}\\
      B_i^vB_i'&=g^{k_{B,i}}B_{i-1}^{k_{E,i}}\enspace\text{for $i=0,\ldots,N-1$}
      &D^vD'&=g^{k_D}
    \end{align*}

  \end{enumerate}
  
\end{protocol}
\end{nicebox}

%%% Local Variables:
%%% TeX-master: "vtm"
%%% End:


\begin{nicebox}
  \begin{protocol}[Commitment-Consistent Proof of a Shuffle]$\quad$\label{prot:ccpos}\\
    \textbf{Common Input.} Generators $g,h_0,\ldots,h_{N-1}\in\Gq$,
    Pedersen commitments $u_0,\ldots,u_{N-1}\in\Gq$, a public key
    $\pk$, elements $w_0,\ldots,w_{N-1}\in\CSpace$ and
    $w_0',\ldots,w_{N-1}'\in\CSpace$.

    \textbf{Private Input.} Exponents
    $r=(r_0,\ldots,r_{N-1})\in\zedq^N$, a permutation
    $\pi\in\permset{N}$, and exponents $s=(s_0,\ldots,s_{N-1})\in\RSpace^N$ such
    that $u_i=g^{r_{\pi(i)}}h_{\pi(i)}$ and
    $w_i'=\Enc_{\pk}(1,s_{\pi^{-1}(i)})w_{\pi^{-1}(i)}$ for $i=0,\ldots,N-1$.

  \begin{enumerate}

  \item $\V$ chooses a seed $s\in\{0,1\}^{\secp}$ randomly, defines
    $e\in\INT{\secpe}^N$ as $e=\PRG(s)$, hands $s$ to $\P$ and
    computes
    \begin{align*}
      A=\prod_{i=0}^{N-1}u_i^{e_i}\enspace.
    \end{align*}

  \item $\P$ chooses $\alpha\in\zedq$,
    $\epsilon_0,\ldots,\epsilon_{N-1}\in\INT{\secpe+\secpv+\secpr}$,
    and $\beta\in\RSpace$ randomly and computes
    \begin{align*}
      A'=g^{\alpha}\prod\nolimits_{i=0}^{N-1}h_i^{\epsilon_i}
      \quad\text{and}\quad B'=\Enc_{\pk}(1,-\beta)\prod\nolimits_{i=0}^{N-1}(w_i')^{\epsilon_i}\enspace.
    \end{align*}
    Then it hands $(A',B')$ to $\V$.

  \item $\V$ chooses $v\in\INT{\secpv}$ randomly and hands $v$ to $\P$.

  \item Let $e_i'=e_{\pi^{-1}(i)}$. $\P$ computes $a=\langle
    r,e'\rangle$, $b=\langle s,e\rangle$, and
    \begin{align*}
      k_A=va+\alpha\enspace,\quad
      k_{B}=vb+\beta \enspace,\quad\text{and}\quad
      k_{E,i}=ve_i'+\epsilon_i\enspace\text{for $i=0,\ldots,N-1$}\enspace.
    \end{align*}
    Then it hands $(k_A,k_B,k_E)$ to $\V$.

  \item $\V$ computes $B=\prod\nolimits_{i=0}^{N-1}w_i^{e_i}$
    and accepts if and only if
    \begin{align*}
      A^vA'&=g^{k_A}\prod\nolimits_{i=0}^{N-1}h_i^{k_{E,i}}
      &B^vB'&=\Enc_{\pk}(1,-k_{B})\prod\nolimits_{i=0}^{N-1}(w_i')^{k_{E,i}}\\
    \end{align*}

  \end{enumerate}
\end{protocol}
\end{nicebox}


\end{document}


%%% Local Variables: 
%%% mode: latex
%%% TeX-master: t
%%% End: 
